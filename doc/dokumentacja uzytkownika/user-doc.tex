\documentclass[a4paper,notitlepage]{article}
\usepackage[a4paper, top=2.5cm, bottom=2.5cm, left=2cm, right=2cm]{geometry}

\usepackage{listings}
\usepackage[utf8]{inputenc}
\usepackage{polski}
\usepackage{amsmath}
\usepackage{alltt}
\usepackage{graphicx}
\usepackage{fancyhdr}
\usepackage[
	bookmarks, 
	colorlinks=true, 
	pdftitle={WikiServer - Dokumentacja Użytkownika}, 
	pdfauthor={Dorota Wąs, Michał Bejda, Michał Sapalski}, 
]{hyperref}
\newlength{\lword}
\newcommand{\define}[3]{
\begin{samepage}
{\settowidth{\lword}{
\textbf{\large #1} = }\vspace{0.2cm}\par\noindent\hangindent=\lword
\textbf{\large #1} = \emph{#2} #3\vspace{0.2cm}\par}
\end{samepage}
}
\lstset{
    language=c++,
    commentstyle=\itshape,
    numbers=left,
    numbersep=5pt,
    frame=single,
    tabsize=2,
    breaklines=true,
    breakatwhitespace=true,
    inputencoding=utf8,
    extendedchars=true,
	texcl=true,
	mathescape=true
}
\begin{document}
\pagestyle{fancy}
\lhead{Dokumentacja Użytkownika --- \textsc{WikiServer}}
\rhead{Dorota Wąs, Michał Bejda, Michał Sapalski}
\tableofcontents
\section{Wymagania}
\begin{itemize}
\item \texttt{Linux} (testowane na \texttt{Ubuntu 10.10} oraz 
	\texttt{Ubuntu 11.04}).
\item \texttt{Python} w wersji co najmniej 2.6 (testowane na 2.6, 2.7)
\item Przeglądarka (testowane na \texttt{Firefox}, \texttt{Chrome})
\end{itemize}

\section{Użytkowanie}
\begin{enumerate}
\item Przed pierwszym uruchomieniem trzeba dodać użytkowników do bazy. 
Można to zrobić programem manage.py 
(\texttt{./manage.py --add-user nick --password hasło}).
Program ten przyjmuje następujące opcje:
\begin{itemize}
    \item \texttt{-h} (lista opcji)
    \item \texttt{-a} (dodaj użytkownika)
    \item \texttt{-r} (usuń użytkownika)
    \item \texttt{-n} (nazwa użytkownika)
    \item \texttt{-p} (hasło użytkownika)
    \item \texttt{-d} (plik z bazą użytkowników, domyślnie users.db)
    \item \texttt{-l} (wypisz wszystkich użytkowników)
\end{itemize}
\item Należy uruchomić server.py (\texttt{./server.py}), który 
będzie serwował pliki. Można go uruchomić z następującymi opcjami:
    \begin{itemize}
    \item \texttt{-h} (lista opcji)
    \item \texttt{-p port} (port dla serwera http, domyślnie 8080)
    \item \texttt{-P port} (port dla serwera https, domyślnie 8081)
    \item \texttt{-c cachedir} (nazwa katalogu w którym serwer może cachować pliki, domyślnie cache)
    \item \texttt{-s cachesize} (rozmiar na tymczasowe pliki w KB, domyślnie 100MB)
    \item \texttt{-d} (plik z bazą użytkowników, domyślnie users.db)
    \end{itemize}
\item Należy uruchomić co najmniej raz hiddenServer.py 
(\texttt{./hiddenServer.py}), który podłączy się do serwera 
(może być na innym komputerze). Można go uruchomić z następującymi opcjami:
    \begin{itemize}
    \item \texttt{-h} (lista opcji)
    \item \texttt{--host HOST} (nazwa serwera, domyślnie localhost)
    \item \texttt{-n name} (nasza nazwa, opcja obowiązkowa)
    \item \texttt{-d dir} (ścieżka do katalogu który chcemy udostępniać)
    \item \texttt{-m mypass} (nasze hasło, domyślnie puste)
    \item \texttt{-p pass} (hasło do przeglądania plików na tym HiddenServerze,
 domyślnie puste)
    \end{itemize}
\item Należy uruchomić przeglądarkę i wpisać jako adres 
\texttt{adres-hosta:8080} (np. \texttt{localhost:8080}).
\item Można też skorzystać z~szyfrowanego połączenia: 
\texttt{https://adres-hosta:8081}.
\item Możemy już przeglądać użytkowników (po podaniu hasła) 
i~pliki przez nich udostępniane (również po podaniu odpowiedniego hasła).
\end{enumerate}

\end{document}
