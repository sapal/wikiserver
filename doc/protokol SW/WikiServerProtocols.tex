\documentclass[a4paper,notitlepage]{article}
\usepackage[a4paper, top=2.5cm, bottom=2.5cm, left=2cm, right=2cm]{geometry}

\usepackage{listings}
\usepackage[utf8]{inputenc}
\usepackage{polski}
\usepackage{amsmath}
\usepackage{alltt}
\usepackage{graphicx}
\usepackage{fancyhdr}
\usepackage[
	bookmarks, 
	colorlinks=true, 
	pdftitle={WikiServer - Protokoły}, 
	pdfauthor={Dorota Wąs, Michał Sapalski}, 
]{hyperref}
\newlength{\lword}
\newcommand{\define}[3]{
\begin{samepage}
{\settowidth{\lword}{
\textbf{\large #1} = }\vspace{0.2cm}\par\noindent\hangindent=\lword
\textbf{\large #1} = \emph{#2} #3\vspace{0.2cm}\par}
\end{samepage}
}
\lstset{
    language=c++,
    commentstyle=\itshape,
    numbers=left,
    numbersep=5pt,
    frame=single,
    tabsize=2,
    breaklines=true,
    breakatwhitespace=true,
    inputencoding=utf8,
    extendedchars=true,
	texcl=true,
	mathescape=true
}
\begin{document}
\pagestyle{fancy}
\lhead{Protocols --- \textsc{WikiServer}}
\rhead{Dorota Wąs, Michał Sapalski}
\tableofcontents
\section{Protokół SW}
Protokół SW (\texttt{Sapalski-Wąs}) służy do komunikacji pomiędzy Hidden Serverem a Serverem. \\
Hidden Server podłącza się do Servera (na porcie 8888) i wysyła polecenie postaci:
\begin{itemize}
    \item \texttt{MYNAMEIS \textbackslash n}
    \item \texttt{username:name \textbackslash n}
    \item \texttt{mypass:pass \textbackslash n}
    \item \texttt{\textbackslash n}
\end{itemize}
Gdzie mypass jest hasłem użytkownika.\\
Połączenie to powinno być przez cały czas podtrzymywane.\\

Server wysyła do Hidden Servera polecenia postaci:
\begin{enumerate}
    \item
\begin{itemize}
    \item \texttt{GET \textbackslash n}
    \item \texttt{filename:sciezka do pliku \textbackslash n}
    \item \texttt{modifytime:data ostatniego uaktualnienia kopii \textbackslash n}
    \item \texttt{id:id zapytania \textbackslash n}
    \item \texttt{originalRequest:oryginalne cialo zapytania \textbackslash n}
    \item \texttt{\textbackslash n}
\end{itemize}
    \item
\begin{itemize}    
    \item \texttt{GOAWAY \textbackslash n}
    \item \texttt{go:adres innego serwera \textbackslash n}
    \item \texttt{\textbackslash n}
\end{itemize}
    \item
\begin{itemize}    
    \item \texttt{REJ \textbackslash n}
    \item \texttt{\textbackslash n}
\end{itemize}
\end{enumerate}
Gdzie wszystkie informacje są stringami bez enterów a pole modifytime jest opcjonalne. \\
Pierwszy odpowiada sytuacji gdy Server prosi Clienta o jakiś plik.\\ 
Drugi odpowiada sytuacji gdy dany Server jest przeciążony i prosi Hidden Server o przełączenie się na inny Server. (Może być wykorzystywane w momencie gdy jest więcej Serverów).\\
Trzeci odpowiada sytuacji gdy podana nazwa użytkownika i/lub hasło są niepoprawne.\\
HS na pierwszy typ zapytania powinien wykonać następujące:
\begin{enumerate}
    \item jeśli hasło (w originalRequest) jest błędne, lub z innego powodu nie ma uprawnien do przeglądania katalogu
    \begin{itemize}   
        \item \texttt{REJ \textbackslash n}
        \item \texttt{username:name \textbackslash n}
        \item \texttt{mypass:pass \textbackslash n}
        \item \texttt{id:id zapytania \textbackslash n}
        \item \texttt{\textbackslash n}
    \end{itemize}
    \item jeśli plik/folder nie istnieje, Hidden Server powinien odpowiedzieć poleceniem postaci:
    \begin{itemize}   
        \item \texttt{NOPE \textbackslash n}
        \item \texttt{username:name \textbackslash n}
        \item \texttt{mypass:pass \textbackslash n}
        \item \texttt{id:id zapytania \textbackslash n}
        \item \texttt{\textbackslash n}
    \end{itemize}
    \item porównać daty modyfikacji pliku 
    \item jeśli data modyfikacji pliku jest wcześniejsza od daty zapisania kopi, Hidden Sever powinien odpowiedzieć poleceniem postaci:
    \begin{itemize}
        \item \texttt{OK \textbackslash n}
        \item \texttt{username:name \textbackslash n}
        \item \texttt{mypass:pass \textbackslash n}
        \item \texttt{id:id zapytania \textbackslash n}
        \item \texttt{\textbackslash n}
    \end{itemize}
    \item w przeciwnym przypadku, powinien odpowiedzieć poleceniem postaci:
    \begin{itemize}
        \item \texttt{OLD \textbackslash n}
        \item \texttt{username:name \textbackslash n}
        \item \texttt{mypass:pass \textbackslash n}
        \item \texttt{id:id zapytania \textbackslash n}
        \item \texttt{size:rozmiar pliku \textbackslash n}
        \item \texttt{filename:sciezka do pliku \textbackslash n}
        \item \texttt{type:typ(directory/file) \textbackslash n}
        \item \texttt{modifytime:czas modyfikacji pliku (float)\textbackslash n}
        \item \texttt{\textbackslash n}
    \end{itemize}
    \item po czym otworzyć nowy kanał do komunikacji z Serverem (na porcie 9999), wysłać w nim dane zgodnie z protokołem PF.
\end{enumerate}
\section{Protokół PF}
Protokół PF (\texttt{Push-File}) służy do przesyłania danych z Hidden Servera do Servera. \\
Hidden Server podłącza się do Servera (na porcie 9999) i wysyła polecenie postaci:
\begin{itemize}
        \item \texttt{PUSH \textbackslash n}
        \item \texttt{username:name \textbackslash n}
        \item \texttt{mypass:pass \textbackslash n}
        \item \texttt{id:id zapytania \textbackslash n}
        \item \texttt{size:rozmiar pliku \textbackslash n}
        \item \texttt{filename:sciezka do pliku \textbackslash n}
        \item \texttt{type:typ(directory/file) \textbackslash n}
        \item \texttt{modifytime:czas modyfikacji pliku (float)\textbackslash n}
        \item \texttt{\textbackslash n}
        \item \texttt{cialo pliku}
    \end{itemize}

\section{Komentarz}
Tworzenie nowego kanału ma na celu polepszenie wydajności, w przypadku gdy Client C1 chce jakiegoś dużego pliku P1 z Hidden Servera HS1, następnie Client C2 chce jakiegoś małego pliku P2 z tego samo Hidden Servera HS1, chcielibyśmy nie musieć czekać aż cały plik P1 zostanie pobrany, gdyż mogłoby to długo potrwać. Zamiast tego chcielibyśmy równolegle pobrać plik P2. \\
Przesyłanie id zapytania w odpowiedzi przez Hidden Server ma na celu rozróżnienie zapytań. W przypadku gdy C1 prosi HS1 o plik P1 a C2 prosi HS1 o plik P2 chcielibyśmy uniknąć sytuacji w której P1 wysyłamy do C2 a P2 wysyłamy do C1. \\
Klient odbierając drugi typ polecenia (\texttt{GOAWAY}), powinien zerwać połączenie z Serverem a następnie nawiązać nowe połączenie z innym Serverem.




   
\end{document}
