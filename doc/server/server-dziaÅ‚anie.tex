\documentclass[a4paper,notitlepage]{article}
\usepackage[a4paper, top=2.5cm, bottom=2.5cm, left=2cm, right=2cm]{geometry}

\usepackage{listings}
\usepackage[utf8]{inputenc}
\usepackage{polski}
\usepackage{amsmath}
\usepackage{alltt}
\usepackage{graphicx}
\usepackage{fancyhdr}
\usepackage[
	bookmarks, 
	colorlinks=true, 
	pdftitle={Projekt na Sieci Komputerowe}, 
	pdfauthor={Dorota Wąs, Michał Sapalski}, 
]{hyperref}
\newlength{\lword}
\newcommand{\define}[3]{
\begin{samepage}
{\settowidth{\lword}{
\textbf{\large #1} = }\vspace{0.2cm}\par\noindent\hangindent=\lword
\textbf{\large #1} = \emph{#2} #3\vspace{0.2cm}\par}
\end{samepage}
}
\lstset{
    language=c++,
    commentstyle=\itshape,
    numbers=left,
    numbersep=5pt,
    frame=single,
    tabsize=2,
    breaklines=true,
    breakatwhitespace=true,
    inputencoding=utf8,
    extendedchars=true,
	texcl=true,
	mathescape=true
}
\begin{document}
\pagestyle{fancy}
\lhead{Sieci Komputerowe --- \textsc{Projekt}}
\rhead{Dorota Wąs, Michał Sapalski}
\tableofcontents
\section{Działanie Servera}
\begin{figure}
\includegraphics[width=\linewidth]{DiagramKlas}
\caption{Diagram klas Servera (nie uwzględniono rzeczy związanych 
z~synchronizacją wątków)}
\end{figure}
\subsection{Start Servera}
Przy starcie \texttt{Server}a uruchamiają się trzy wątki: 
\begin{itemize}
\item \texttt{HSServer}
\item \texttt{HttpServer}
\item \texttt{PushFileServer}
\end{itemize}
Tworzona jest także klasa FileManager.

\subsection{Obsługa połączenia z~HiddenServerem}
Każde połączenie z~\texttt{HiddenServer}em jest obsługiwane przez 
dwa wątki w klasie \texttt{HiddenServerConnection}. Jeden z~nich służy 
do~wysyłania zapytań, drugi do przetwarzania odpowiedzi.
Jak tylko \texttt{HiddenServer} prześle \texttt{MYNAMEIS},
połączenie jest rejestrowane w~\texttt{FileManager.hiddenServerConnections}.

\subsection{Obsługa połączenia HTTP}
Każde połączenie z~użytkownikiem jest obsługiwane przez osobny wątek: 
\texttt{HttpConnection}. Przetwarza on~zapytanie HTTP na~żądanie konkretnego
pliku i~wywołuje \texttt{FileManager.getFileInfo}. W~zależności od~typu 
zwróconego pliku (\texttt{FileInfo.fileType}) zwraca on~użytkownikowi błąd 
(\texttt{"not found"}), listing katalogu (\texttt{"directory"}) lub plik.
Odczytywanie pliku/listingu katalogu jest takie samo:
otwiera się plik \texttt{FileInfo.filename}, czyta się tyle bajtów, ile jest
dostępnych (\texttt{FileInfo.currentSize}), wysyła użytkownikowi a~następnie 
czeka na~\texttt{FileInfo.fileModified} aż~więcej bajtów będzie dostępnych. 
Jak cały plik zostanie wysłany, kończy się proces.

\subsection{Zdobywanie pliku przez FileManagera}
Gdy wywoływana jest metoda \texttt{FileManager.getFileInfo} 
wykonywane są następujące czynności:
\begin{itemize}
\item Zapytaniu nadawane jest kolejne \texttt{id}
\item Do odpowiedniego \texttt{HiddenServerConnection} wysyłane jest zapytanie 
o plik
\item Wątek czeka na odpowiednim \texttt{threading.Condition} na odpowiedź
\item Po uzyskaniu odpowiedzi zwraca odpowiednie \texttt{FileInfo}
\end{itemize}

\subsection{Obsługa zapytania przez HiddenServerConnection}
Zapytania do~\texttt{HiddenServera} są~kolejkowane i~wysyłane
asynchronicznie z~przetwarzaniem odpowiedzi. 
\texttt{Hid\-den\-Server\-Connection}
trzyma kolejkę wysłanych zapytań. Kiedy przyjdzie odpowiedź wywołuje 
\texttt{File\-Manager\-.process\-Response} oraz \texttt{answerCondition.notify}.

\subsection{Obsługa odpowiedzi przez FileManagera}
Gdy wywoływane jest metoda \texttt{FileManager.processResponse}
wykonywane są następujące czynności:
\begin{itemize}
\item W~przypadku odpowiedzi \texttt{DIRECTORY} lub \texttt{OLD} tworzone jest 
nowe \texttt{FileInfo}
\item Odpowiednie \texttt{FileInfo} jest mapowane do~odpowiedniego 
\texttt{id}.
\end{itemize}

\subsection{Przesyłanie plików z~HiddenServera}
Każde połączenie przychodzące z~\texttt{HiddenServer}a jest obsługiwane przez
osobny wątek: \texttt{PushFileConnection}. Zapisuje on otrzymane dane do~pliku
wskazywanego przez \texttt{FileManager.fileInfo}.

\subsection{Zarządzanie cachem}
Zarządzanie cachem odbywa się za pomocą metody \texttt{FileManager.cleanCache},
która jest wywoływana przy każdym \texttt{FileManager.processResponse}.
Każdy obiekt \texttt{FileInfo} trzyma liczbę aktualnie używających go~procesów,
usunięcie go z~cache'u może nastąpić tylko, gdy żaden proces go~nie używa.

Pliki są~wybierane do~usunięcia biorąc pod uwagę następujące wielkości:
\begin{itemize}
\item Data ostatniego dostępu do pliku
\item Data modyfikacji pliku
\item Liczba dostępów do pliku
\end{itemize}

\subsection{TODO}
Dla uproszczenia, następujące rzeczy nie zostały tu uwzględnione:
\begin{itemize}
\item Metody synchronizacji wątków
\item Watchdogi (procesy obsługujące połączenia powinny być przerywane
w~przypadku braku odpowiedzi przez dłuższy czas)
\end{itemize}
\end{document}
