\documentclass[a4paper,notitlepage]{article}
\usepackage[a4paper, top=2.5cm, bottom=2.5cm, left=2cm, right=2cm]{geometry}

\usepackage{listings}
\usepackage[utf8]{inputenc}
\usepackage{polski}
\usepackage{amsmath}
\usepackage{alltt}
\usepackage{graphicx}
\usepackage{fancyhdr}
\usepackage[
	bookmarks, 
	colorlinks=true, 
	pdftitle={Projekt na Sieci Komputerowe}, 
	pdfauthor={Dorota Wąs, Michał Bejda, Michał Sapalski}, 
]{hyperref}
\newlength{\lword}
\newcommand{\define}[3]{
\begin{samepage}
{\settowidth{\lword}{
\textbf{\large #1} = }\vspace{0.2cm}\par\noindent\hangindent=\lword
\textbf{\large #1} = \emph{#2} #3\vspace{0.2cm}\par}
\end{samepage}
}
\lstset{
    language=c++,
    commentstyle=\itshape,
    numbers=left,
    numbersep=5pt,
    frame=single,
    tabsize=2,
    breaklines=true,
    breakatwhitespace=true,
    inputencoding=utf8,
    extendedchars=true,
	texcl=true,
	mathescape=true
}
\begin{document}
\pagestyle{fancy}
\lhead{Sieci Komputerowe --- \textsc{Projekt}}
\rhead{Dorota Wąs, Michał Bejda, Michał Sapalski}
\tableofcontents
\section{Działanie Servera}
\begin{figure}
\includegraphics[width=\linewidth]{DiagramKlas}
\caption{Diagram klas Servera (nieistotne metody/pola pominięte)}
\end{figure}
\subsection{Start Servera}
Przy starcie \texttt{Server}a uruchamiają się trzy wątki: 
\begin{itemize}
\item \texttt{HSServer}
\item \texttt{HttpServer}
\item \texttt{PushFileServer}
\end{itemize}
Tworzona jest także klasa FileManager.

\subsection{Obsługa połączenia z~HiddenServerem}
Każde połączenie z~\texttt{HiddenServer}em jest obsługiwane przez 
dwa wątki w klasie \texttt{HiddenServerConnection}. Jeden z~nich służy 
do~wysyłania zapytań, drugi do przetwarzania odpowiedzi.
Jak tylko \texttt{HiddenServer} prześle \texttt{MYNAMEIS},
połączenie jest rejestrowane w~\texttt{FileManager.hiddenServerConnections}.

\subsection{Obsługa połączenia HTTP}
Każde połączenie z~użytkownikiem jest obsługiwane przez osobny wątek: 
\texttt{HttpConnection}. Przetwarza on~zapytanie HTTP na~żądanie konkretnego
pliku i~wywołuje \texttt{FileManager.getFileInfo}. W~zależności od~typu 
zwróconego pliku (\texttt{FileInfo.fileType}) zwraca on~użytkownikowi błąd 
(\texttt{"not found"}), prośbę o~podanie hasła 
(\texttt{"authentication required"}), 
listing katalogu (\texttt{"directory"}) lub plik.
Odczytywanie pliku/listingu katalogu jest takie samo:
otwiera się plik \texttt{FileInfo.filename}, czyta się tyle bajtów, ile jest
dostępnych (\texttt{FileInfo.currentSize}), wysyła użytkownikowi a~następnie 
czeka na~\texttt{FileInfo.fileModified} aż~więcej bajtów będzie dostępnych. 
Jak cały plik zostanie wysłany, kończy się wątek.

\subsection{Zdobywanie pliku przez FileManagera}
Gdy wywoływana jest metoda \texttt{FileManager.getFileInfo} 
wykonywane są następujące czynności:
\begin{itemize}
\item Zapytaniu nadawane jest kolejne \texttt{id}
\item Do odpowiedniego \texttt{HiddenServerConnection} wysyłane jest zapytanie 
o plik
\item Wątek czeka na odpowiednim \texttt{threading.Condition} na odpowiedź
\item Po uzyskaniu odpowiedzi zwraca odpowiednie \texttt{FileInfo}
\end{itemize}

\subsection{Obsługa zapytania przez HiddenServerConnection}
Zapytania do~\texttt{HiddenServera} są~kolejkowane i~wysyłane
asynchronicznie z~przetwarzaniem odpowiedzi. 
\texttt{Hid\-den\-Server\-Connection}
trzyma kolejkę wysłanych zapytań. Kiedy przyjdzie odpowiedź wywołuje 
\texttt{File\-Manager\-.process\-Response} oraz \texttt{answerCondition.notify}.

\subsection{Obsługa odpowiedzi przez FileManagera}
Gdy wywoływane jest metoda \texttt{FileManager.processResponse}
wykonywane są następujące czynności:
\begin{itemize}
\item W~przypadku odpowiedzi \texttt{OLD} tworzone jest 
nowe \texttt{FileInfo}
\item Odpowiednie \texttt{FileInfo} jest mapowane do~odpowiedniego 
\texttt{id}.
\end{itemize}

\subsection{Przesyłanie plików z~HiddenServera}
Każde połączenie przychodzące z~\texttt{HiddenServer}a jest obsługiwane przez
osobny wątek: \texttt{PushFileConnection}. Zapisuje on otrzymane dane do~pliku
wskazywanego przez \texttt{FileManager.fileInfo}.

\subsection{Zarządzanie cachem}
Zarządzanie cachem odbywa się za pomocą metody \texttt{FileManager.cleanCache},
która jest wywoływana przy każdym \texttt{FileManager.processResponse}.
Każdy obiekt \texttt{FileInfo} trzyma liczbę aktualnie używających go~wątków,
usunięcie go z~cache'u może nastąpić tylko, gdy żaden wątek go~nie używa.

Pliki są~wybierane do~usunięcia biorąc pod uwagę następujące wielkości:
\begin{itemize}
\item Data ostatniego dostępu do pliku
\item Data modyfikacji pliku
\item Liczba dostępów do pliku
\end{itemize}

Przy kończeniu programu cache jest opróżniane.

\subsection{Obsługa błędów}
W~przypadku wystąpienia błędu z~połączeniem, wątek je~obsługujący jest
zabijany. Szczególnym przypadkiem jest błąd w~\texttt{PushFileConnection}
--- wtedy plik jest dodatkowo oznaczany jako popsuty, żeby pozostałe 
wątki, które go~używają mogły się o~tym dowiedzieć.

\subsection{Synchronizacja wątków}
Poza opisanymi tu~metodami synchronizacji, w~miejscach, w~których jest 
to~potrzebne wątki używają \texttt{RLock}ów, żeby zapewnić spójność
danych.

\section{Działanie Klienta (Hidden Servera)}
\subsection{Start Klienta}
Klienta podłącza się do Hidden Servera i rejestruje się poprzez \texttt{MYNAMEIS}.
\subsection{Obsługa zapytań}
Jeśli klient dostaje zapytanie typu \texttt{GET} obsługuje je w następujący sposób.
\begin{itemize}
\item Sprawdza czy plik/folder o zadanej ścieżce istnieje (jeśli nie to wysyła odpowiedź \texttt{NOPE}).
\item Sprawdza czy zapytanie jest o plik, jeśli tak:
    \begin{itemize}
    \item Porównuje datę ostatniej modyfikacji, otrzymaną w zapytaniu. Jeśli plik jest aktualny to wysyła odpowiedź \texttt{OK}.
    \item W przeciwny przypadku wysyła odpowiedź \texttt{OLD}. Po czym tworzony jest osobny wątek \texttt{newThreadPushFile} który łączy się z Serverem i wysyła odpowiedni plik.
    \end{itemize}
\item Sprawdza czy zapytanie jest o katalog, jeśli tak:
    \begin{itemize}
    \item Tworzy tymczasowy plik w katalogu \texttt{/tmp/}, do którego zapisuje wynik wylistowania katalogu.
    \item Wysyła odpowiedź \texttt{OLD}. Po czym tworzony jest osobny wątek \texttt{newThreadPushFile} który łączy się z Serverem i wysyła odpowiedni plik.
    \end{itemize}
\end{itemize}

\end{document}
