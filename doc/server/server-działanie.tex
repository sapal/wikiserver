\documentclass[a4paper,notitlepage]{article}
\usepackage[a4paper, top=2.5cm, bottom=2.5cm, left=2cm, right=2cm]{geometry}

\usepackage{listings}
\usepackage[utf8]{inputenc}
\usepackage{polski}
\usepackage{amsmath}
\usepackage{alltt}
\usepackage{graphicx}
\usepackage{fancyhdr}
\usepackage[
	bookmarks, 
	colorlinks=true, 
	pdftitle={Projekt na Sieci Komputerowe}, 
	pdfauthor={Dorota Wąs, Michał Sapalski}, 
]{hyperref}
\newlength{\lword}
\newcommand{\define}[3]{
\begin{samepage}
{\settowidth{\lword}{
\textbf{\large #1} = }\vspace{0.2cm}\par\noindent\hangindent=\lword
\textbf{\large #1} = \emph{#2} #3\vspace{0.2cm}\par}
\end{samepage}
}
\lstset{
    language=c++,
    commentstyle=\itshape,
    numbers=left,
    numbersep=5pt,
    frame=single,
    tabsize=2,
    breaklines=true,
    breakatwhitespace=true,
    inputencoding=utf8,
    extendedchars=true,
	texcl=true,
	mathescape=true
}
\begin{document}
\pagestyle{fancy}
\lhead{Sieci Komputerowe --- \textsc{Projekt}}
\rhead{Dorota Wąs, Michał Sapalski}
\tableofcontents
\section{Działanie Servera}
\begin{figure}
\includegraphics[width=\linewidth]{DiagramKlas}
\caption{Diagram klas Servera}
\end{figure}
\subsection{Start Servera}
Przy starcie \verb!Server!a uruchamiają się trzy wątki: 
\begin{itemize}
\item \verb!HSServer!
\item \verb!HttpServer!
\item \verb!PushFileServer!
\end{itemize}
Tworzona jest także klasa FileManager.

\subsection{Obsługa połączenia z~HiddenServerem}
Każde połączenie z~\verb!HiddenServer!em jest obsługiwane przez 
dwa wątki w klasie \verb!HiddenServerConnection!. Jeden z~nich służy 
do~wysyłania zapytań, drugi do przetwarzania odpowiedzi.
Jak tylko \verb!HiddenServer! prześle \verb!MYNAMEIS!,
połączenie jest rejestrowane w~\verb!FileManager.hiddenServerConnections!.

\subsection{Obsługa połączenia HTTP}
Każde połączenie z~użytkownikiem jest obsługiwane przez osobny wątek: 
\verb!HttpConnection!. Przetwarza on~zapytanie HTTP na~żądanie konkretnego
pliku i~wywołuje \verb!FileManager.getFileInfo!. W~zależności od~typu zwróconego
pliku (\verb!FileInfo.fileType!) zwraca on~użytkownikowi błąd 
(\verb!"not found"!), listing katalogu (\verb!"directory"!) lub plik.
Odczytywanie pliku/listingu katalogu jest takie samo:
otwiera się plik \verb!FileInfo.filename!, czyta się tyle bajtów, ile jest
dostępnych (\verb!FileInfo.currentSize!), wysyła użytkownikowi a~następnie czeka
na~\verb!FileInfo.fileModified! aż~więcej bajtów będzie dostępnych. Jak cały 
plik zostanie wysłany, kończy się proces.

\subsection{Zdobywanie pliku przez FileManagera}
Gdy wywoływana jest metoda \verb!FileManager.getFileInfo! 
wykonywane są następujące czynności:
\begin{itemize}
\item Zapytaniu nadawane jest kolejne \verb!id!
\item Do odpowiedniego \verb!HiddenServerConnection! wysyłane jest zapytanie 
o plik
\item Wątek czeka na odpowiednim \verb!threading.Condition! na odpowiedź
\item Po uzyskaniu odpowiedzi zwraca odpowiednie \verb!FileInfo!
\end{itemize}

\subsection{Obsługa zapytania przez HiddenServerConnection}
Zapytania do~\verb!HiddenServera! są~kolejkowane i~wysyłane
asynchronicznie z~przetwarzaniem odpowiedzi. \verb!HiddenServerConnection!
trzyma kolejkę wysłanych zapytań. Kiedy przyjdzie odpowiedź wywołuje 
\verb!FileManager.processResponse! oraz \verb!answerCondition.notify!.

\subsection{Obsługa odpowiedzi przez FileManagera}
Gdy wywoływane jest metoda \verb!FileManager.processResponse!
wykonywane są następujące czynności:
\begin{itemize}
\item W~przypadku odpowiedzi \verb!DIRECTORY! lub \verb!OLD! tworzone jest nowe
\verb!FileInfo!
\item Odpowiednie \verb!FileInfo! jest mapowane do~odpowiedniego 
\verb!requestId!.
\end{itemize}

\subsection{Przesyłanie plików z~HiddenServera}
Każde połączenie przychodzące z~\verb!HiddenServer!a jest obsługiwane przez
osobny wątek: \verb!PushFileConnection!. Zapisuje on otrzymane dane do~pliku
wskazywanego przez \verb!FileManager.fileInfo!.

\subsection{TODO}
Dla uproszczenia, następujące rzeczy nie zostały tu uwzględnione:
\begin{itemize}
\item Zarządzanie cachem (kasowanie starych plików)
\item Mechanizm blokowania plików w cache (nie można usunąć pliku w~cache 
jeśli jakiś proces jeszcze go~potrzebuje)
\item Watchdogi (procesy obsługujące połączenia powinny być przerywane
w~przypadku braku odpowiedzi przez dłuższy czas)
\end{itemize}
\end{document}
