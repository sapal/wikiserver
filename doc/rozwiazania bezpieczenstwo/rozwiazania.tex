\documentclass[a4paper,notitlepage]{article}
\usepackage[a4paper, top=3.5cm, bottom=3.5cm]{geometry}


\usepackage{listings}
\usepackage[utf8]{inputenc}
\usepackage{polski}
\usepackage{amsmath}
\usepackage{alltt}
\usepackage{graphicx}
%\usepackage{tikz}
\usepackage{fancyhdr}
%\usetikzlibrary{arrows,snakes,automata,shapes.geometric,shapes.symbols,shapes.callouts,calc}
%\usepackage{tkz-2d}
\usepackage[
	bookmarks, 
	colorlinks=true, 
	pdftitle={Rozwiązania problemów związanych z bezpieczeństwem 
	w Wikiserverze}, 
	pdfauthor={Dorota Wąs, Michał Bejda, Michał Sapalski}, 
]{hyperref}
\newlength{\lword}
\newcommand{\define}[3]{
\begin{samepage}
{\settowidth{\lword}{\textbf{\large #1} = }\vspace{0.2cm}\par\noindent\hangindent=\lword
\textbf{\large #1} = \emph{#2} #3\vspace{0.2cm}\par}
\end{samepage}
}
\lstset{
    language=c++,
    commentstyle=\itshape,
    numbers=left,
    numbersep=5pt,
    frame=single,
    tabsize=2,
    breaklines=true,
    breakatwhitespace=true,
    inputencoding=utf8,
    %extendedchars=true,
	%texcl=true,
	%mathescape=true
}
\lstset{
    inputencoding=utf8,
    literate={ą}{{\k{a}}}1
             {Ą}{{\k{A}}}1
             {ę}{{\k{e}}}1
             {Ę}{{\k{E}}}1
             {ó}{{\'o}}1
             {Ó}{{\'O}}1
             {ś}{{\'s}}1
             {Ś}{{\'S}}1
             {ł}{{\l{}}}1
             {Ł}{{\L{}}}1
             {ż}{{\.z}}1
             {Ż}{{\.Z}}1
             {ź}{{\'z}}1
             {Ź}{{\'Z}}1
             {ć}{{\'c}}1
             {Ć}{{\'C}}1
             {ń}{{\'n}}1
             {Ń}{{\'N}}1
}
\begin{document}
\pagestyle{fancy}
\lhead{\textsc{Wikiserver} --- Zastosowane rozwiązania}
\rhead{Dorota Wąs, Michał Bejda, Michał Sapalski}

\section{Rozwiązania problemów związanych z~bezpieczeństwem}
\subsection{Zagrożenia poufności}
\subsubsection{Dostęp do nieudostępnionych plików na HiddenServerze}
Zaimplementowaliśmy proponowane rozwiązanie (patrz: \emph{,,Analiza
bezpieczeństwa Wikiservera''} 2.1 \emph{Dostęp do nieudostępnionych
plików na~HiddenServerze} sekcja \emph{Sposoby zabezpieczania}). 

HiddenServer sprawdza, czy plik jest udostępniony zanim zwróci go 
w~odpowiedzi.

\subsubsection{Podmiana Servera / podsłuchiwanie połączenia}
Zaimplementowano proponowane rozwiązanie (patrz: \emph{,,Analiza
bezpieczeństwa Wikiservera''} 2.3 \emph{Podmiana Servera / podsłuchiwanie 
połączenia}
sekcja \emph{,,Sposoby zabezpieczania''}) przy wykorzystaniu protokołu
SSL. W~celu całkowitego zabezpieczenia przed atakiem typu 
\emph{man-in-the-middle} certyfikaty autentykacji Servera są dystrybuowane
razem z~programem.

\subsubsection{Ograniczanie dostępu do plików}
Zarówno przeglądanie plików na HiddenServerach, jak i~wyświetlanie
listy dostępnych HiddenServerów wymaga podania poprawnego loginu
i~hasła. Zastosowane zostało rozwiązanie, w~którym hasło jest 
sprawdzane na HiddenServerze (patrz: \emph{,,Analiza
bezpieczeństwa Wikiservera''} 3.2 \emph{Ograniczanie dostępu do plików}
sekcja \emph{,,Proponowane rozwiązanie''}).
\subsubsection{Maskowanie ściąganych plików}

\subsection{Zagrożenia integralności}
\subsubsection{Zmienianie danych ,,w~locie''}
\subsubsection{Podmiana HiddenServera}
\label{podmianaHS}
Zalogowanie do Servera wymaga podania loginu i~hasła. Błąd związany
z~dwukrotnym logowaniem tego samego użytkownika (patrz: \emph{,,Analiza
bezpieczeństwa Wikiservera''} 4.2 \emph{Podmiana HiddenServera}). 

W~bazie danych Servera są przechowywane jedynie sumy SHA512 haseł,
żeby utrudnić ich złamanie w~przypadku gdy baza danych dostanie
się w~niepowołane ręce.
\subsubsection{Atak przez odgrywanie}

\subsection{Zagrożenia dostępności}
\subsubsection{Podmiana HiddenServera}
Patrz: \emph{,,Analiza
bezpieczeństwa Wikiservera''} 5.1 \emph{Podmiana HiddenServera} oraz 
\ref{podmianaHS}.
\subsubsection{Kopie zapasowe bazy danych}
Kopia zapasowa pliku, w~którym znajduje się baza danych powinna być
robiona regularnie. Zwykle Server będzie uruchomiony na serwerze, którego
administrator robi kopie zapasowe danych na dysku lub w~inny sposób
zabezpiecza przed ich utratą (np. stosując macierz RAID). W~takim przypadku
przywrócenie poprzedniego stanu Servera w~razie awarii jest bardzo proste.
Gdy Server jest uruchamiany na komputerze nie wyposażonym w~żadne systemy
zapewniające ochronę przed utratą plików, trzeba zadbać o~to samemu.

\end{document}
