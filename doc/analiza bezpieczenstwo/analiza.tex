\documentclass[a4paper,notitlepage]{article}
\usepackage[a4paper, top=3.5cm, bottom=3.5cm]{geometry}

\usepackage{listings}
\usepackage[utf8]{inputenc}
\usepackage{polski}
\usepackage{amsmath}
\usepackage{alltt}
\usepackage{graphicx}
%\usepackage{tikz}
\usepackage{fancyhdr}
%\usetikzlibrary{arrows,snakes,automata,shapes.geometric,shapes.symbols,shapes.callouts,calc}
%\usepackage{tkz-2d}
\usepackage[
	bookmarks, 
	colorlinks=true, 
	pdftitle={Analiza bezpieczeństwa Wikiservera}, 
	pdfauthor={Dorota Wąs, Michał Sapalski}, 
]{hyperref}
\newlength{\lword}
\newcommand{\define}[3]{
\begin{samepage}
{\settowidth{\lword}{\textbf{\large #1} = }\vspace{0.2cm}\par\noindent\hangindent=\lword
\textbf{\large #1} = \emph{#2} #3\vspace{0.2cm}\par}
\end{samepage}
}
\lstset{
    language=c++,
    commentstyle=\itshape,
    numbers=left,
    numbersep=5pt,
    frame=single,
    tabsize=2,
    breaklines=true,
    breakatwhitespace=true,
    inputencoding=utf8,
    extendedchars=true,
	texcl=true,
	mathescape=true
}
\begin{document}
\pagestyle{fancy}
\lhead{\textsc{Wikiserver} --- analiza bezpieczeństwa}
\rhead{Dorota Wąs, Michał Sapalski}

\section{Konwencja w~przykładach ataków}
W~podanych dalej przykładach będziemy przyjmować następującą konwencję:
\begin{itemize}
\item Server działa pod adresem \texttt{http://hogwart/} 
z~limitem wielkości cache równym 100MB,
\item Harry ma uruchomiony HiddenServer o nazwie \texttt{harry} 
zarejestrowany na Serverze \texttt{hogwart},
serwujący pliki z~katalogu \texttt{/home/harry/public/},
\item Ron przegląda pliki z~\texttt{http://hogwart/harry},
\item Voldemort chce w~jakiś sposób naruszyć bezpieczeństwo systemu.
\end{itemize}

\section{Zagrożenia poufności --- bieżąca implementacja}
Wikiserver został napisany z~założeniem, że każdy użytkownik (Client) może
wyświetlać pliki dowolnego HiddenServera. Ponadto Server jest uważany 
za zaufany. Jednak w~rzeczywistych zastosowaniach nie zawsze tak jest, co
może prowadzić do zagrożenia bezpieczeństwa w~rozmaitych obszarach.

Warto zauważyć, że nawet gdy te założenia są spełnione, to nadal
jest wiele sposobów przeprowadzenia ataku. 

\subsection{Dostęp do nieudostępnionych plików na HiddenServerze}
\subsubsection*{Opis zagrożenia}
Atakujący może dostać się do praktycznie dowolnego pliku na HiddenServerze,
nie tylko do udostępnionych plików.
\subsubsection*{Przykładowy atak}
\begin{itemize}
\item Voldemort wykonuje zapytanie 
\texttt{http://hogwart/harry/../secret\_file}
\item Server przekazuje do Harry'ego zapytanie o~\texttt{../secret\_file}
\item Harry odpowiada na zapytanie zwracając plik 
\texttt{/home/harry/secret\_file}, który nie był udostępniony
\end{itemize}

\subsubsection*{Sposoby zabezpieczenia}
HiddenServer powinien sprawdzać, czy plik jest udostępniony zanim
zwróci go w~odpowiedzi na zapytanie.

\subsection{Przepełnienie pamięci cache}
\label{przepelnienieCacheu}
\subsubsection*{Opis zagrożenia}
Jest możliwe, że nastąpi przepełnienie pamięci cache na Serverze (czyli,
że folder \texttt{cache} będzie zajmował więcej miejsca niż określa to 
konfiguracja).
\subsubsection*{Przykładowy atak}
\begin{itemize}
\item Voldemort rejestruje się na Serverze \texttt{hogwart} pod nazwą
\texttt{voldemort} udostępniając pliki z~katalogu \texttt{/home/voldemort/evil}
\item Voldemort umieszcza duży plik \texttt{big\_file} (np. 1GB) 
w~katalogu \texttt{/home/voldemort/evil}
\item Voldemort wykonuje zapytanie \texttt{http://hogwart/voldemort/big\_file}
\item Server próbuje cache'ować plik \texttt{big\_file}, co powoduje że
użycie dysku wzrasta do 1GB.
\end{itemize}

\subsubsection*{Sposoby zabezpieczania}
Server nie powinien próbować cache'ować plików większych niż maksymalny
rozmiar cache'u. Warto zauważyć, że Server powinien wymuszać, że 
HiddenServer poda poprawną wielkość pliku poprzez zakończenie połączenia
gdy liczba przesłanych bajtów przekroczy podaną przez HiddenServer wielkość.

\subsection{Podmiana Servera / podsłuchiwanie połączenia}
\label{podmianaServera}
\subsubsection*{Opis zagrożenia}
Jednym z~założeń jest, że Server jest uważany za zaufany. Jest to istotne,
gdyż ma on dostęp do informacji o~tym kto ściągał
jakie pliki oraz z~którego HiddenServera. Poprzez podmianę Servera lub
podsłuchiwanie połączeń atakujący może przejąć te informacje.
\subsubsection*{Przykładowy atak}
\begin{itemize}
\item Voldemort przechwytuje pakiety do Servera i~zapamiętuje
jaki użytkownik ściąga jaki plik.
\end{itemize}
\subsubsection*{Sposoby zabezpieczania}
Zarówno połączenie między Serverem a~HiddenServerem jak i~Clientem
a~Serverem powinno być szyfrowane. Trzeba również zaimplementować
autentykację Servera --- na przykład za pomocą szyfrowania z~kluczem
publicznym (zakładamy, że HiddenServer i~Client mają dostęp
do klucza publicznego Servera).

\section{Zagrożenia poufności --- implementacja Wikiservera z~kontrolą 
dostępu do plików}
Gdy przyjmiemy założenie, że niektóre pliki powinny być dostępne jedynie
dla wybranych Clientów, oraz że Server może być przejęty przez atakującego,
to musimy poradzić sobie z~dodatkowymi problemami dotyczącymi poufności.

\subsection{Maskowanie ściąganych plików}
\subsubsection*{Bieżąca implementacja}
W obecnej implementacji Server ma wszystkie informacje o~tym kto ściągał
jakie pliki oraz z~którego HiddenServera. 
\subsubsection*{Proponowane rozwiązanie}
Możliwe są dwa rozwiązania:
\begin{itemize}
\item Połączenie Client -- HiddenServer jest szyfrowane. Server ma 
informacje, że Client łączył się z~HiddenServerem, ale nie wie
jaki plik ściągał.
\item Przy ściąganiu Client prosi o~kilka plików na~raz, ale wykorzystując
szyfrowanie z~kluczem publicznym zapewniamy, że jest w~stanie odczytać
tylko jeden z~nich. Server nie ma pewnej informacji, że Client ściągnął
jakiś plik, ani że w~ogóle otrzymał jakieś informacje od konkretnego 
HiddenServera.
\end{itemize}
\subsubsection*{Wady}
Zarówno w~jednym, jak i w~drugim rozwiązaniu cache'owanie plików przez
Server jest niemożliwe.


\subsection{Ograniczanie dostępu do plików}
\subsubsection*{Bieżąca implementacja}
W obecnej implementacji wszystkie pliki udostępniane przez HiddenServer 
są publiczne. Zmniejsza to funkcjonalność Wikiservera, gdyż uniemożliwia
dzielenie się plikami, do których nie każdy powinien mieć dostęp.

\subsubsection*{Proponowane rozwiązanie}
Możliwe są dwa rozwiązania:
\begin{itemize}
\item Client wysyła informacje uwierzytelniające do Servera, który sprawdza
czy ma on dostęp do konkretnego pliku.
\item Client wysyła informacje uwierzytelniające do HiddenServera 
(za pośrednictwem Servera). Sprawdzanie uprawnień odbywa się na HiddenServerze.
\end{itemize}
\subsubsection*{Uwagi}
Drugie rozwiązanie jest lepsze, ze względu na to, że uprawnienia 
nie muszą być znane Serverowi. Warto jednak zauważyć, że w~prostej
implementacji drugiego rozwiązania jest pewien problem, który nie występuje
w~rozwiązaniu pierwszym: atakujący jest w~stanie rozróżnić, czy HiddenServer
jest niepodłączony do Servera, czy tylko nie ma do niego dostępu --- 
w~drugim przypadku czas odpowiedzi Servera jest dłuższy, gdyż musi
się komunikować z~HiddenServerem.

Warto również zaznaczyć, że konieczna jest autentykacja HiddenServerów 
(opisane w~(\ref{podmianaHiddenServera})). W~przypadku gdy jej nie ma, 
jest możliwy następujący atak:
\begin{itemize}
\item Harry nie jest podłączony do Servera
\item Voldemort łączy się z~Serverem jako \texttt{harry} i~udostępnia
jakieś pliki
\item Ron wchodzi na \texttt{http://hogwart/harry} i~podaje dane
uwierzytelniające
\item Server przesyła te dane do Voldemorta, żeby sprawdził czy są poprawne
\item Voldemort wylogowuje się z~Servera
\item Harry łączy się z~Serverem
\item Voldemort używa danych uwierzytelniających, które dostał od Rona aby
uzyskać dostęp do plików na \texttt{http://hogwart/harry}
\end{itemize}

\subsection{Podsłuchiwanie połączenia}
\subsubsection*{Opis zagrożenia}
Ponieważ żadne z~połączeń nie jest szyfrowane, atakujący może je podsłuchać,
co daje mu możliwość uzyskania bezprawnego dostępu do plików.
\subsubsection*{Proponowane rozwiązanie}
Wszystkie połączenia powinny być szyfrowane.

\section{Zagrożenia integralności}
\subsection{Zmienianie danych ,,w~locie''}
\subsubsection*{Opis zagrożenia}
Przy nieszyfrowanym połączeniu atakujący może dowolnie zmieniać przesyłane
dane.
\subsubsection*{Proponowane rozwiązanie}
Wszystkie połączenia powinny być szyfrowane.

\subsection{Podmienienie HiddenServera}
\subsubsection*{Opis zagrożenia}
Atakujący może połączyć się z~Serverem używając nie swojej nazwy użytkownika.
Client łączący się z~takim HiddenServerem dostanie inne pliki 
niż się spodziewał.

Warto zauważyć, że w~bieżącej implementacji taki atak jest możliwy nawet
jeżeli użytkownik jest już zalogowany (drugie logowanie ,,anuluje'' pierwsze).
\subsubsection*{Proponowane rozwiązanie}
HiddenServery powinny być autentykowane przy logowaniu 
(opisane w~(\ref{podmianaHiddenServera}))

\section{Zagrożenia dostępności}
\subsection{Podmiana HiddenServera}
\label{podmianaHiddenServera}
\subsubsection*{Opis zagrożenia}
Atakujący może połączyć się z~Serverem używając nie swojej nazwy użytkownika.
Client nie będzie w~stanie dostać się do plików zaatakowanego HiddenServera.

\subsubsection*{Proponowane rozwiązanie}
HiddenServery powinny być autentykowane przy logowaniu (na przykład przez
podanie loginu i~hasła). Warto zauważyć, że przy logowaniu musi być
autentykacja Servera oraz szyfrowanie połączenia 
(patrz: (\ref{podmianaServera})), gdyż w~przeciwnym
przypadku atakujący może wykraść hasło HiddenServera.

\subsection{Zapchanie pamięci cache}
Patrz: Przepełnienie pamięci cache (\ref{przepelnienieCacheu}). 
\subsubsection*{Opis zagrożenia}
Atakujący może spowodować zapchanie pamięci cache niepotrzebnymi plikami
wykonując wiele zapytań o~losowe pliki z~HiddenServerów. Warto zauważyć,
że atakujący nie musi ściągać pliku na swój komputer --- Server będzie
kontynuował pobieranie do pamięci cache nawet po przerwaniu połączenia
przez Clienta.

Zapchanie pamięci cache powoduje zmniejszenie wydajności działania Servera,
a~co za tym idzie, może spowodować brak dostępności dla Clientów.

\subsubsection*{Proponowane rozwiązanie}
Server może przerywać pobieranie pliku z~HiddenServera w~momencie gdy
Client zrywa połączenie. Ponadto, Server może tworzyć listy Clientów,
którzy często zrywają połączenie i~nie obsługiwać ich (albo np. zmniejszać
im przepustowość).

\end{document}
