\documentclass[a4paper,notitlepage]{article}
\usepackage[a4paper, top=2.5cm, bottom=2.5cm, left=2cm, right=2cm]{geometry}

\usepackage{listings}
\usepackage[utf8]{inputenc}
\usepackage{polski}
\usepackage{amsmath}
\usepackage{alltt}
\usepackage{graphicx}
\usepackage{fancyhdr}
\usepackage[
	bookmarks, 
	colorlinks=true, 
	pdftitle={Projekt na Sieci Komputerowe}, 
	pdfauthor={Dorota Wąs, Michał Sapalski}, 
]{hyperref}
\newlength{\lword}
\newcommand{\define}[3]{
\begin{samepage}
{\settowidth{\lword}{
\textbf{\large #1} = }\vspace{0.2cm}\par\noindent\hangindent=\lword
\textbf{\large #1} = \emph{#2} #3\vspace{0.2cm}\par}
\end{samepage}
}
\lstset{
    language=c++,
    commentstyle=\itshape,
    numbers=left,
    numbersep=5pt,
    frame=single,
    tabsize=2,
    breaklines=true,
    breakatwhitespace=true,
    inputencoding=utf8,
    extendedchars=true,
	texcl=true,
	mathescape=true
}
\begin{document}
\pagestyle{fancy}
\lhead{Sieci Komputerowe --- \textsc{Projekt}}
\rhead{Dorota Wąs, Michał Sapalski}
\section*{Opis maszyn wirtualnych}
Maszyny wirtualne tworzą prostą sieć:
\begin{itemize}
\item Puppy --- Server jest podłączony na zewnątrz (poprzez eth0:192.168.254.134 -- NAT),
oraz z~pozostałymi maszynami wirtualnymi (eth1:192.168.10.1). 
Będzie na nim uruchomiony Server.
\item Puppy --- HiddenServer 1 nie ma połączenia na zewnątrz, a jedynie 
z~pozostałymi maszynami wirtualnymi (eth0:192.168.10.2).
\item Puppy --- HiddenServer 2 nie ma połączenia na zewnątrz, a jedynie 
z~pozostałymi maszynami wirtualnymi (eth0:192.168.10.3).
\end{itemize}
Aby uruchomić projekt, wystarczy uruchomić już wpisane polecenia w~konsoli: 
najpierw na Serverze, a~potem na~HiddenServerach.

HiddenServer 1 będzie serwował pliki z~\verb!/usr/share/pixmaps!,
a~HiddenServer 2 --- z~\verb!~/Web-Server!. Można sprawdzić, że~wikiserver 
działa łącząc się z~dowolnej maszyny z~Serverem na porcie 8080. 
Przykładowo, z~maszyny gospodarza: \verb!http://192.168.254.134:8080/!.
\end{document}
